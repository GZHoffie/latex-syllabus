%<*empire>
\begin{itemize}
\item	When was the German Empire established and under what circumstances was this done?
\item	Which states were included and which ones were excluded? Explain the ``Kleindeutsche Lösung'' vs. the ``Gro\ss deutsche Lösung''.
\item	What was the system of government?
\item	How did Germany develop economically and politically during the empire?
\item	How did the empire come to an end?
\item	...
\end{itemize}
%</empire>
%<*colonies>
\begin{itemize}
\item	When and how were the German colonies established? In which parts of the world? What was the reason for doing so?
\item	Describe the German colonial history in China.
\item	Describe the ongoing political controversy about the German colonies in Africa, specifically in what is now Namibia.  What are the points of contention between the relevant groups in Namibia and the German government?
\item	Today, Germany has no colonies. How did the colonial empire come to an end?
\item	...
\end{itemize}
%</colonies>
%<*weimar>
\begin{itemize}
\item	When and how was the Weimar Republic established? Explain the system of government used and how it differed from the previous system. Discuss its rise and fall in the context of the first World War.
\item	What made it special for Germany? Compare with other systems of government in Europe at the time.
\item	Explain the challenges faced (e.g., hyperinflation, world economic crisis, political murders) and how they contributed to the end of the republic, but don't go into detail regarding the actual fall.
\item	...
\end{itemize}
%</weimar>
%<*hitler>
\begin{itemize}

\item	On March $24^{\text{th}}$, 1933, the parliament of the Weimar Republic passed the \textit{Enabling Act}, which changed the constitution and gave Germany's new Chancellor, Adolph Hitler, unprecedented powers.
\item	Describe the contents of the Enabling Act. What laws and regulations did it propose to change?
\item	Discuss the events surrounding the Enabling Act. What was the context of the introduction of the Act? What did Hitler's government promise and what reasons did the government give to parliament as to why the act was necessary?
\item	Which parties and parliamentarians approved the Act? Which ones were opposed? Give some prominent examples.
\item	What were the actual consequences of the passage of the Act? In what way is it fair to say that the passage signified the end of the Weimar Republic?
\item	...
\end{itemize}
%</hitler>
%<*holocaust>
\begin{itemize}
\item	The National Socialist government under Hitler quickly started to introduce laws and regulations persecuting Jews. Explain the reasoning that was given by the government for this. How did society react? What effect did these laws have on the Jewish population?
\item	Explain the significance of the \textit{Reichskristallnacht} (``Night of broken glass'').
\item	Before and during the Second World War, the persecution of Jews intensified and millions were killed in concentration camps and extermination camps. Give numbers and particularly egregious examples (there are sadly more than enough!).
\item	How was the situation dealt with after the war?
\item	... (there is more than enough to say!)
\end{itemize}
%</holocaust>
%<*brdddr>
\begin{itemize}
\item	When and under what circumstances were the Federal Republic of Germany (BRD) and the German Democratic Republic (DDR) founded?
\item	What was the status of Berlin until 1990? What events led to the so-called ``Berlin airlift''? Explain the circumstances of this event for Germany and the significance for the German people.
\item	What were the national holidays of the BRD and the DDR before 1990? What events did they commemorate?
\item	... (there are many other aspects; choose at least one)
\end{itemize}
%</brdddr>
%<*wall>
\begin{itemize}
\item	When was the Berlin Wall built? What were the reasons for building it? What was the political context at the time?
\item	The Wall became a symbol for the division of Germany and the closed border between the two countries. Many people nevertheless attempted to flee. Those that failed were often killed or imprisoned, while those that succeeded were treated as heroes. Recount some of the well-known episodes involving failed and successful attempts to cross the wall and border.
\item	What was the final trigger for the ``collapse of the wall'' on September $9^{\text{th}}$, 1989?
\item	What events had been leading up to the collapse in the previous months/years?
\item	...
\end{itemize}
%</wall>
%<*boarder>
\begin{itemize}
\item	What happened in 2016 when the ``borders were opened''? What was the situation in Europe and Germany?
\item	What was the attitude and reaction of the various other European states?
\item	How many refugees entered Germany in 2015 and the following years? Where did they come from?
\item	What is the attitude of the political parties?
\item	What is the current situation of the refugees?
\item	What are the possible implications for German society? What other people of non-German origins now live in Germany?
\item	...
\end{itemize}
%</boarder>
%<*elections>
\begin{itemize}
\item	Explain the bicameral system and role of the chancellor and the president of Germany. How are they elected?
\item	How does the popular vote for the \textit{Bundestag} work? What are the \textit{Erststimme} and \textit{Zweitstimme}? How many voting districts are there? How often does an election take place?
\item	 What is the 5\% hurdle? What is an \textit{Überhangsmandat}?
\item	...
\end{itemize}
%</elections>
%<*parties>
\begin{itemize}
\item	Summarize the origins, the history, the traditional ideology and the contemporary goals and influence of the following parties:
\begin{itemize}
    \item SPD
    \item CDU/CSU
    \item The Greens
    \item Die Linke
    \item AfD
\end{itemize}
\item	...
\end{itemize}
%</parties>
%<*reform>
\begin{itemize}
\item	How and when were the reforms decided?
\item	What were the changes (e.g., \textit{Arbeitslosengeld I & II, Sozialhilfe})?
\item	What is the current national and international opinion on these reforms? What are the respective attitudes of the German political parties?
\end{itemize}
%</reform>
%<*economy>
\begin{itemize}
\item	What are the main sectors of the German economy (agriculture, manufacturing, services, finance etc.) and their relative importance?
\item	Explain the concept of the German \textit{Mittelstand}. What sort of companies comprise it? What is the relative importance of these mid-size companies compared with major companies?
\item	What is the current state of the economy? How is the situation in the job market?
\item	What are the current, major challenges for the economy?
\item	...
\end{itemize}
%</economy>
%<*surplus>
\begin{itemize}
\item	How is the current account surplus calculated?
\item	How large is it with the USA? With China? With other countries and regions?
\item	Is the surplus considered harmful to the German economy? To the US economy?
\item	What measures might be implemented to reduce the surplus?
\item	What are the respective German and US attitude towards the German surplus?
\item	...
\end{itemize}
%</surplus>
%<*car>
\begin{itemize}
\item	Present the history of the car, from its invention to the present day, in the context of Germany.
\item	Briefly present the major German car companies and discuss their influence on and role in society in the $20^{\text{th}}$ century and today. In particular, consider the following aspects:
\begin{itemize}
    \item What was the role of the major manufacturers during the reign of the national socialists from 1933 to 1945? Have these manufacturers faced their past behavior? If so, when and in what way did they do this?
    \item What is the current role of the car in German society? How many cars do people have, on average? What percentage of travel is by car, as opposed to mass transit (for short distances) and other forms of transport? What types of cars do people buy in Germany? Compare with the US and China.
    \item What is the importance of car manufacturing (including associated parts manufacturers) for the German economy? How many people are employed in the car industry? How does is this reflected, for example, in the current \textit{Diesel scandal} and the reactions of the political parties and government agencies to the crisis?
\end{itemize}
\item	...
\end{itemize}
%</car>
%<*school>
\begin{itemize}
\item	Explain the 3-element school system (\textit{Hauptschule}, \textit{Realschule}, \textit{Gymnasium}). How many years are needed to complete the respective elements?
\item	How, when and by whom is the decision made to which of these elements a child will go?
\item	After the \textit{Realschule}, students may start vocational training (\textit{Lehre}) in a system that combines school classes with on-the-job training. Explain briefly how this works.
\item	In theory, the system is supposed to place children in schools according to their academic aptitudes. In practice, the system is criticized for being influenced by the social background of the children's families: poor families are less likely to have their child go to a \textit{Gymnasium}. Research and present the contemporary discussion on these issues.
\item	...
\end{itemize}
%</school>
%<*abitur>
\begin{itemize}
\item	12 vs. 13 years of school. When was this changed? What is the situation today?
\item	What subjects are needed to complete the \textit{Abitur}? How is the grade calculated?
\item	What is a \textit{Leistungskurs}?
\item	Summarize the discussions surrounding the ,,\textit{Zentralabitur}``? Which steps are being taken now?
\item	Does the \textit{Abitur} allow the taking of any course at any university? What is a \textit{Numerus Clausus (NC)}?
\item	Compare this university admissions system with that of other countries, including China and the US.
\item	...
\end{itemize}
%</abitur>
%<*schoolabitur>
\begin{itemize}
\item	Explain the 3-element school system (\textit{Hauptschule}, \textit{Realschule}, \textit{Gymnasium}). How many years are needed to complete the respective elements?
\item	How, when and by whom is the decision made to which of these elements a child will go?
\item	After the \textit{Realschule}, students may start vocational training (\textit{Lehre}) in a system that combines school classes with on-the-job training. Explain briefly how this works.
\item	In theory, the system is supposed to place children in schools according to their academic aptitudes. In practice, the system is criticized for being influenced by the social background of the children's families: poor families are less likely to have their child go to a \textit{Gymnasium}. Research and present the contemporary discussion on these issues.
\item	12 vs. 13 years of school. When was this changed? What is the situation today?
\item	What subjects are needed to complete the \textit{Abitur}? How is the grade calculated?
\item	What is a \textit{Leistungskurs}?
\item	Summarize the discussions surrounding the ,,\textit{Zentralabitur}``? Which steps are being taken now?
\item	Does the \textit{Abitur} allow the taking of any course at any university? What is a \textit{Numerus Clausus (NC)}?
\item	Compare this university admissions system with that of other countries, including China and the US.
\item	...
\end{itemize}
%</schoolabitur>
%<*tertiary>
\begin{itemize}
\item	After completing school, there are a variety of options for tertiary education, including studying at a traditional university, a ``university of applied sciences'' (\textit{Fachhochschule}), or completing a dual education system of apprenticeship and studies, \textit{Duales Ausbildungssystem}. Describe these options and their differences to each other and to systems in other countries in detail.
\item	Comment on the status of tertiary education in German society: what percentage of people opt for apprenticeships, university studies etc.? How have these numbers changed historically and how do they compare with other countries in Europe and the US?
\item	There is also an ongoing discussion about social mobility: if the parents of a young person studied at university, that person is much more likely to go to university themselves. Similarly, people from a poor or immigrant background face challenges entering university. Describe the issues using the most recent data you can find and compare with other countries.
\item	...
\end{itemize}
%</tertiary>

%<*ludwig>
\begin{itemize}
\item	The historical background when Ludwig governed Bavaria;
His family and beginning years of a young king;
...
\item	Construction of \textit{Neuschwanstein} Castle:

How long did it take to build this castle?
Why Ludwig decided to build it?
Any stories during the construction?
What about other buildings funded by Ludwig?
(With intro about \textit{Neuschwanstein} Castle, artwork inside it)
...
\item	Ludwig's contribution to art history

His friendship with Richard Wagner;
Why did a king keep such a long run friendship with an artist?
In what aspect did Wagner influence Ludwig?
Intro about Wagner's later career sponsored by Ludwig
...
\item	Ludwig's last years

What's the international situation and what's Bavaria's position in it?
How did Ludwig get deposed and what's the influence of it?
Any myth left unsolved by Ludwig?
...
\item	Ludwig in today's popular culture

Two films (1972 and 2012) and a famous Musical Ludwig II;
Neuschwanstein Castle in Fussen and Bavaria's tourism today
\end{itemize}
%</ludwig>
%<*music>
\begin{itemize}
\item	Why we always talk about German and Austrian music together
\item	German musicians from baroque to romanticism
\item	The three B's: Bach, Beethoven and Brahms
\item	German musical instrument innovation: woodwind and the Boehm system
\item	Berlin Philharmonic Orchestra
\end{itemize}
%</music>
%<*stasi>
\begin{itemize}
\item	How and when was the Stasi created?
\item	What was the remit of the Stasi? What made it different from other public / state security agencies?
\item	What was the role of the IM (Informelle Mitarbeiter)? How many where there?
\item	When did the Stasi finally fall? What happened to the files?
\item	What happened to the leaders of the Stasi after the fall of the wall and reunification? What about the other members and the IM?
\item	What is the current status of the Stasi files?
\end{itemize}
%</stasi>
%<*philosophy>
\begin{itemize}
\item	Choose one or more renowned German or Austrian philosophers (e.g., Kant, Hegel, Wittgenstein, etc.) and describe that person's work, life and times.
\item ...
\end{itemize}
%</philosophy>
%<*holidays>
\begin{itemize}
\item	What public holidays are there in Germany? Are they the same for every state?
\item	Describe the traditions surrounding some selected public and unofficial holidays, such as Carnival, Father's day / Mother's day, Easter, St. Nichloas's day, Pfingsten etc.
\item	...
\end{itemize}
%</holidays>
%<*beer>
\begin{itemize}
\item	What types of beer are there in Germany (there are at least 5 categories, including light and dark wheat beer, pilsner, etc.)
\item	Describe the various brewing methods – go into technical detail!
\item	The German Reinheitsgebot – describe origin, history and tradition. How was it influenced by the EU?
\item	Describe differences to other beer traditions in Europe (Belgium, Netherlands, France, Scotland, Ireland)) and beer types (Stout, IPA, etc.)
\item	...
\end{itemize}
%</beer>
%<*food>
\begin{itemize}
\item	Describe in detail the various food specialties in the northern parts of Germany (e.g., Matjes, Rollmops, Sauerkraut, etc.) Give their history and origins.
\item	Describe in detail the various food specialties in the southern parts of Germany (e.g., Spätzle, Schweinshaxe, Saumagen, Weisswurst etc.) Give their history and origins.
\item	...
\end{itemize}
%</food>
%<*elisabeth>
\begin{itemize}
\item	Origin
\item	Life and ``beauty cult''
\item	Death
\item	Political context
\item	Cultural influence in the 20th century
\item	Musical: Elisabeth
\item	...
\end{itemize}
%</elisabeth>
